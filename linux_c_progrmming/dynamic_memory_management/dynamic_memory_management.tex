% Created 2022-07-17 Sun 12:59
% Intended LaTeX compiler: pdflatex
\documentclass[11pt]{article}
\usepackage[utf8]{inputenc}
\usepackage[T1]{fontenc}
\usepackage{graphicx}
\usepackage{longtable}
\usepackage{wrapfig}
\usepackage{rotating}
\usepackage[normalem]{ulem}
\usepackage{amsmath}
\usepackage{amssymb}
\usepackage{capt-of}
\usepackage{hyperref}
\author{mrc20}
\date{\today}
\title{}
\hypersetup{
 pdfauthor={mrc20},
 pdftitle={},
 pdfkeywords={},
 pdfsubject={},
 pdfcreator={Emacs 28.1 (Org mode 9.5.2)}, 
 pdflang={English}}
\begin{document}

\tableofcontents

\#+title Dynamic Memory Management
\section{Linux下进程的划分}
\label{sec:org3ddd6d9}
Linux下的应用程序编译后生成的可执行文件(未调入到内存执行前)分为代码段(.text)、数据段(.data)、
和未初始化数据段(.bss),这三段分别存储指令机器码,初始化过的全局变量和静态局部变量,未初始化过的
全局变量和静态局部变量;局部变量所占空间和malloc申请的空间不占该文件的空间
\section{堆和栈的区别}
\label{sec:org787fb1f}
\section{动态内存的申请和释放}
\label{sec:orgb30539e}
\section{常见的动态内存错误}
\label{sec:org2438242}
\end{document}