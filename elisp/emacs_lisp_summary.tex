% Created 2023-01-29 Sun 00:25
% Intended LaTeX compiler: pdflatex
\documentclass[11pt]{article}
\usepackage[utf8]{inputenc}
\usepackage[T1]{fontenc}
\usepackage{graphicx}
\usepackage{longtable}
\usepackage{wrapfig}
\usepackage{rotating}
\usepackage[normalem]{ulem}
\usepackage{amsmath}
\usepackage{amssymb}
\usepackage{capt-of}
\usepackage{hyperref}
\author{dclinee}
\date{\textit{<2023-01-28 Sat>}}
\title{50-point summary of Emacs Lisp}
\hypersetup{
 pdfauthor={dclinee},
 pdftitle={50-point summary of Emacs Lisp},
 pdfkeywords={},
 pdfsubject={},
 pdfcreator={Emacs 28.1 (Org mode 9.5.2)}, 
 pdflang={English}}
\begin{document}

\maketitle
\tableofcontents

\section{1.Basics}
\label{sec:org676a3e9}
\subsection{A. "set" assigns a value to a symbol}
\label{sec:org8d4cb6e}
\begin{verbatim}
(set 'fname "Dclinee")
\end{verbatim}
Since quoting the symbol name is so common a second assignment opetator
was created, "setq".
\subsection{B. "setq" sets a value of a variable}
\label{sec:org04359fa}
\begin{verbatim}
(setq fname "dclinee")
\end{verbatim}
but can take multiple arguments
\begin{verbatim}
(setq first_name "denng"
      secod_name "changlin")
\end{verbatim}
It returns the last value set.
\end{document}